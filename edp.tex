%        File: edp.tex
%     Created: lun sep 17 05:00  2018 C
% Last Change: lun sep 17 05:00  2018 C
%
\documentclass[12pt,a4paper]{book}
\usepackage[utf8]{inputenc}
\usepackage[spanish, es-noquoting]{babel}
\usepackage[left=2.5cm,right=2.5cm,top=2.5cm,bottom=2.5cm]{geometry}
\usepackage{amsmath}
\usepackage{amsfonts}
\usepackage{amssymb}
\usepackage{amsthm}
\usepackage{mathtools}
\usepackage{tikz,tikz-cd}
\usetikzlibrary{arrows, babel}
\usepackage{url}
\usepackage[colorlinks=true,linktocpage=true,pagebackref=true,linkcolor=blue]{hyperref}
\usepackage{graphicx}

\DeclarePairedDelimiter\norm{\lVert}{\rVert}
\DeclarePairedDelimiter\esc{\langle}{\rangle}

%Fuente Palatino:
%\usepackage[sc]{mathpazo}
%Fuente Times:
%\usepackage{newtxtext}
%\usepackage{newtxmath}
%Fuente Libertine:
%\usepackage{libertine}
%\usepackage[libertine]{newtxmath}

\newtheorem{thm}{Teorema}[section]
\newtheorem{prop}[thm]{Proposición}
\newtheorem{lema}{Lema}
\newtheorem{corol}[thm]{Corolario}
\theoremstyle{definition} \newtheorem{defn}[thm]{Definición}
\theoremstyle{definition} \newtheorem{ejemplo}[thm]{Ejemplo}
\theoremstyle{definition} \newtheorem{ejercicio}[thm]{Ejercicio}
\theoremstyle{remark} \newtheorem*{obs}{Observación}

\newcommand{\RR}{\mathbb{R}}
\newcommand{\NN}{\mathbb{N}}
\newcommand{\cte}{\mathrm{cte.}}
\newcommand{\eps}{\varepsilon}

\title{Apuntes de Ecuaciones en Derivadas Parciales}
\author{Guillermo Gallego Sánchez}
\date{}

\begin{document}
\maketitle
\tableofcontents
\chapter{Derivada débil y espacios de Sobolev}
\begin{ejemplo}\label{ejemplo1}
  Supongamos que se nos da un problema de contorno, como por ejemplo
  \begin{equation}
    \begin{cases}
      -y''+2y=e^x+\cos x \\
      y(0)=1, y(1)=2.
    \end{cases}
    \label{eq:ejemplo1}
  \end{equation}
  Es fácil definir lo que entendemos como una solución del problema~\eqref{eq:ejemplo1}: una función $y\in C^2([0,1])$ que cumpla la ecuación y los datos. Sin embargo, podemos cambiar la función a la derecha de la ecuación por otra con una forma más complicada, por ejemplo, que no sea continua, como
  \begin{equation*}
    f(x)=
    \begin{cases}
      1, & x \in [0,1/2) \\
      0, & x \in [1/2,1).
    \end{cases}
  \end{equation*}
  De modo que ahora queremos resolver el problema de contorno
  \begin{equation}
    \begin{cases}
      -y''+2y=f(x) \\
      y(0)=1, y(1)=2.
    \end{cases}
    \label{eq:ejemplo2}
  \end{equation}
  En este caso ya no podemos pedir que la solución sea $C^2$, en esta sección vamos a tratar de ver en qué espacios pueden vivir estas funciones que podemos entender como ``soluciones'' de problemas como el~\eqref{eq:ejemplo2}. 

  En primer lugar, vamos a cambiar ligeramente el aspecto de nuestra ecuación. Para ello consideramos una función $\varphi$ en $[0,1]$, con $\varphi(0)=\varphi(1)=0$ tan buena como queramos, por ejemplo $\varphi\in C^{\infty}([0,1])$ y la multiplicamos por la ecuación
  \begin{equation*}
    -y''\varphi + 2y \varphi = f \varphi.
  \end{equation*}
  Integramos a ambos lados y obtenemos
  \begin{equation*}
    -\int_0^1 y'' \varphi + \int_0^1 2y \varphi = \int_0^1 f \varphi.
  \end{equation*}
  Integrando por partes
  \begin{equation*}
    \int_0^1 y' \varphi' + 2 \int_0^1 y \varphi = \int_0^1 f \varphi.
  \end{equation*}
  Buscaremos entonces qué funciones $y$ pueden hacer que estas integrales tengan sentido. Los espacios en los que viven estas $y$ serán los que llamaremos \emph{espacios de Sobolev}. \qed
\end{ejemplo}

\section{Repaso de espacios $L^p$}

A partir de ahora, $\Omega\subset \RR^N$ será un conjunto abierto y conexo, es decir, un dominio.
\begin{defn}
  Dado $p\geq 1$, se llama \emph{espacio $L^p$} en $\Omega$ al conjunto
  \begin{equation*}
    L^p(\Omega)=\left\{ f:U\rightarrow \RR \text{ medibles} : \int_{\Omega} |f|^p < \infty \right\}.
  \end{equation*}
  Se define la \emph{norma-$p$} de una función $f\in L^p(\Omega)$ como el número
  \begin{equation*}
    \norm{f}_{p}=\left( \int_{\Omega}|f|^p \right)^{1/p}.   
  \end{equation*}
\end{defn}

\begin{prop}[Desigualdad de Hölder]
  Sean $p,q$ tales que 
  \begin{equation*}
    \frac{1}{p}+\frac{1}{q}=1
  \end{equation*}
  y sean
  $f\in L^p(\Omega)$ y $g\in L^p(\Omega)$. Entonces se verifica la siguiente desigualdad
  \begin{equation*}
    \int_{\Omega}|f| |g| \leq \norm{f}_p \norm{g}_q.
  \end{equation*}
\end{prop}

\begin{proof}
  Para probar la desigualdad de Hölder haremos uso del siguiente lema:
  \begin{lema}\label{desigualdad}
    Sean $a\geq b \geq 0$, y $\lambda \in (0,1)$. Se verifica la siguiente desigualdad
    \begin{equation*}
      a^\lambda b^{1-\lambda} \leq \lambda a + (1-\lambda) b.
    \end{equation*}
  \end{lema}

  \begin{proof}(del Lema~\ref{desigualdad})
    Si $a=0$ ó $b=0$, entonces el resultado es trivial. Supongamos entonces que $a,b > 0$. En tal caso, podemos considerar $x=a/b$ y queremos probar la desigualdad
    \begin{equation*}
      x^{\lambda} \leq \lambda x + (1-\lambda).
    \end{equation*}
    Para verlo, consideremos la función 
    \begin{equation*}
      g(x)=x^{\lambda}-\lambda x - (1-\lambda).
    \end{equation*}
    Si tomamos su derivada, tenemos
    \begin{equation*}
      g'(x)=\lambda (x^{\lambda -1}-1).
    \end{equation*}

    Ahora, si $a \geq b$, tenemos que $x\geq 1$ y, como $\lambda\in (0,1)$, $\lambda -1 \leq 0$ y $x^{\lambda -1}\leq 1$. Por tanto, $g'(x)<0$, y, como $g(1)=0$; para todo $x\geq 1$, $g(x)\leq 0$.
  \end{proof}

  Volviendo a la demostración de la desigualdad de Hölder, llamemos $a=|f|^p/\norm{f}_p^p$ y $b=|g|^q/\norm{g}_q^q$. Supongamos sin pérdida de generalidad que $a \geq b$ y llamemos $\lambda = 1/p$, de modo que $1-\lambda=1/q$. Tenemos entonces, aplicando el lema
  \begin{equation*}
    \left( \frac{|f|^p}{\norm{f}_p^p} \right)^{1/p} \left( \frac{|g|^q}{\norm{g}^q_q} \right)^{1/q}\leq\frac{1}{p}\frac{|f|^p}{\norm{f}_p^p}+\frac{1}{q}\frac{|f|^q}{\norm{f}_q^q}.
  \end{equation*}
  Integrando todo en $\Omega$, queda
  \begin{equation*}
    \int_{\Omega}\frac{|f||g|}{\norm{f}_p \norm{g}_q}\leq\frac{1}{p}\int_\Omega \frac{|f|^p}{\norm{f}_p^p}+\frac{1}{q}\int_\Omega \frac{|f|^q}{\norm{f}_q^q}=\frac{1}{p}+\frac{1}{q}=1.
  \end{equation*}
  Por tanto
  \begin{equation*}
    \int_\Omega |f||g| \leq \norm{f}_p \norm{g}_q,
  \end{equation*}
  tal y como queríamos probar.
\end{proof}

\begin{prop}
  Si $\Omega$ es acotado, entonces se tienen las siguientes inclusiones
  \begin{equation*}
    L^1(\Omega) \supset L^2(\Omega) \supset L^3(\Omega) \supset \cdots \supset L^{p-1}(\Omega) \supset L^{p}(\Omega) \supset L^{p+1}(\Omega) \supset \cdots
  \end{equation*}
\end{prop}

\begin{proof}
  En efecto, si $1\leq p < q$, por la desigualdad de Hölder
  \begin{equation*}
    \int_{\Omega} |f|^p=\int_{\Omega} |f^p| = \int_{\Omega} |f^p| |\chi_\Omega| \leq \norm{f^p}_q \norm{\chi_\Omega}_r,
  \end{equation*}
  con $r$ tal que
  \begin{equation*}
    \frac{1}{q}+\frac{1}{r}=1.
  \end{equation*}

  Por tanto, 
  \begin{equation*}
    \norm{f}_p^p=\int_{\Omega} |f|^p\leq \norm{f^p}_q \norm{\chi_\Omega}_r =\norm{f}_q^p \mu(\Omega)^{1/r}.
  \end{equation*}
  Luego
  \begin{equation*}
    \norm{f}_p \leq \cte \norm{f}_q.
  \end{equation*}
  
\end{proof}

\begin{prop}
  El espacio $L^p(\Omega)$ equipado con la norma-$p$ es un espacio normado.
\end{prop}
\begin{proof}
  Lo único no trivial que hay que demostrar es la desigualdad triangular, esto es
  \begin{equation*}
    \norm{f+g}_p \leq \norm{f}_p + \norm{g}_p.
  \end{equation*}
  Veámosla
  \begin{align*}
    \norm{f+g}_p^p &= \int_\Omega |f+g|^p = \int_\Omega |f+g||f+g|^{p-1} \leq \int_\Omega (|f|+|g|) |f+g|^{p-1} \\ & = \int_\Omega |f| |f+g|^{p-1} + \int_\Omega |g| |f+g|^{p-1} \leq \norm{f}_p \norm{|f+g|^{p-1}}_q +\norm{g}_p \norm{|f+g|^{p-1}}_q
  \end{align*}
  Donde $q$ es el necesario para que se cumpla la desigualdad de Hölder, es decir, 
  \begin{equation*}
    \frac{1}{q}+\frac{1}{p}=1.
  \end{equation*}
  Despejando $q$, tenemos que vale precisamente $q=p/(p-1)$.

  Pero entonces
  \begin{equation*}
    \norm{|f+g|^{p-1}}_q=\left( \int_\Omega (|f+g|^{p-1})^q \right)^{1/q}=\left( \int_\Omega (|f+g|^p) \right)^{(p-1)/p}=\norm{f+g}_p^{p-1}.
  \end{equation*}
  
  Por tanto,
  \begin{equation*}
    \norm{f+g}_p^p \leq \norm{f+g}_p^{p-1}(\norm{f}_p + \norm{g}_p).
  \end{equation*}
  Dividiendo a ambos lados por $\norm{f+g}_p^{p-1}$ tenemos lo que se quería probar.
\end{proof}

Que los espacios $L^p$ sean normados me va a permitir hablar de convergencia de funciones en $L^p$. Así, si $(f_n)$ es una sucesión en $L^p(\Omega)$ y $f\in L^p(\Omega)$, diremos que la sucesión $(f_n)$ \emph{converge a $f$ en $L^p$} y se denota
\begin{equation*}
  f_n \rightarrow f \ L^p
\end{equation*}
si y sólo si $\lim_{n\rightarrow \infty} \norm{f_n-f}_p=0$.

Nos interesará ahora recordar ciertos resultados de teoría de la medida que nos relacionen integrales y convergencias y que nos serán útiles más adelante. Pasamos a enunciarlos a continuación, sin demostración.
\begin{thm}[Teorema de la convergencia monótona]
  Sea una sucesión $(f_n)$ de funciones $f_n \in L^1(\Omega)$ tales que
  \begin{equation*}
    f_1 \leq f_2 \leq \cdots f_n \leq f_{n+1} \leq \cdots.
  \end{equation*}
  Entonces 
  \begin{equation*}
    \int_\Omega \lim_{n\rightarrow \infty } |f_n| = \lim_{n\rightarrow \infty} \int_\Omega |f_n|.
  \end{equation*}
\end{thm}

\begin{thm}[Teorema de la convergencia dominada]
  Sea una sucesión $(f_n)$ de funciones $f_n \in L^1(\Omega)$ tales que existe una función $g\in L^1$ tal que $|f_n|\leq g $ para cada $n\in \NN$ y existe una función $f$ tal que $f_n \rightarrow f$ en casi todo punto. Entonces
  \begin{equation*}
    \lim_{n\rightarrow \infty} \int_\Omega |f_n| = \int_\Omega |f|.
  \end{equation*}
\end{thm}

\begin{thm}[Lema de Fatou]
  Sea una sucesión $(f_n)$ de funciones $f_n \in L^1$ tales que $f_n \geq 0$ para cada $n\in \NN$. Entonces 
  \begin{equation*}
  \int_\Omega \liminf_{n\rightarrow \infty} |f_n| \leq  \liminf_{n\rightarrow \infty}\int_\Omega  |f_n|. 
  \end{equation*}
\end{thm}

Haciendo uso de alguno de estos resultados, vamos a ver que los espacios $L^p$, con sus respectivas normas-$p$, son completos.

\begin{prop}
  El espacio $L^p(\Omega)$ es de Banach. Además, si $p=2$, podemos definir el producto
  \begin{equation*}
    \esc{u,v}=\int_{\Omega}uv,
  \end{equation*}
  que dota a $L^2(\Omega)$ de la estructura de espacio de Hilbert.
\end{prop}
\begin{proof}
  Supongamos que $(f_n)$ es una sucesión de Cauchy en $L^p$. Sin pérdida de generalidad, podemos tomar una subsucesión tal que $\norm{f_{n+1}-f_n}_p\leq 2^{-n}$. Tenemos que encontrar una función $f\in L^p$ tal que $\lim_{n\rightarrow \infty} \norm{f_n-f}_p=0$. 

  Considero entonces la sucesión de funciones $(g_n)$, con
  \begin{equation*}
    g_n = |f_1|+|f_2-f_1|+|f_3-f_2|+\cdots+ |f_n-f_{n-1}|.
  \end{equation*}
  Esta sucesión es claramente monótona creciente. Además,
  \begin{equation*}
    \norm{g_n}_p\leq \norm{f_1}_p + \sum_{i=1}^{n-1}\norm{f_{i+1}-f_i}_p \leq \norm{f_1}_p + \sum_{i=1}^{\infty}\norm{f_{i+1}-f_i}_p \leq \norm{f_1}_p \sum_{i=1}^{\infty}2^{-i}= \cte
  \end{equation*}
  De modo que las $g_n \in L^p(\Omega)$. Pero entonces las $g_n^p \in L^1(\Omega)$, ya que 
 \begin{equation*}
 \int_\Omega |g_n^p| \leq \int_\Omega |g_n|^p \leq \infty.   
  \end{equation*}
  Podemos aplicar entonces el teorema de la convergencia monótona para obtener que
  \begin{equation*}
    \int_{\Omega} \lim_{n\rightarrow \infty} |g_n|^p=\int_{\Omega} \lim_{n\rightarrow \infty} |g_n^p| = \lim_{n\rightarrow \infty} \int_{\Omega} |g_n^p|= \lim_{n\rightarrow \infty} \int_{\Omega} |g_n|^p.
  \end{equation*}
  Tenemos entonces que si $g=\lim_{n\rightarrow \infty}g_n$, 
  \begin{equation*}
    \lim_{n\rightarrow \infty} \norm{g_n-g}_p = 0,
  \end{equation*}
o, lo que es lo mismo
\begin{equation*}
  g_n \rightarrow g \ L^p.
\end{equation*}

Para ver que $(f_n)$ converge en $L^p$, podemos usar que la sucesión es de Cauchy para observar que entonces debe converger puntualmente a una función $f$ en casi todo punto. Además,
\begin{equation*}
  |f_n|=|f_1|+\sum_{i=1}^n(|f_{i+1}|-|f_i|) \leq |f_1| + \sum_{i=1}^n |f_{i+1}-f_i| = g_n \leq g.
\end{equation*}
Ahora, como elevar a $p$ es continuo, tenemos que $|f_n^p | \leq g^p \in L^1(\Omega)$ y que $f_n^p \rightarrow f^p$ en casi todo punto. Por tanto, $|f_n-f|^p\leq 2 |g|^p$ y $f_n^p-f^p \rightarrow 0$ en casi todo punto. Aplicando el teorema de la convergencia dominada, tenemos que
  \begin{equation*}
    \lim_{n\rightarrow \infty} \int_{\Omega} |f_n-f|^p= 0.
  \end{equation*}
  Luego
  \begin{equation*}
    \lim_{n\rightarrow \infty}\norm{f_n-f}_p = 0,
  \end{equation*}
  o, lo que es lo mismo
  \begin{equation*}
    f_n \rightarrow f \ L^p.
  \end{equation*}
\end{proof}

En lo que sigue, haremos bastante uso del siguiente teorema, que no demostraremos:
\begin{prop}\label{densidad1}
  El conjunto $C_c(\Omega)$ de las funciones continuas con soporte compacto en $\Omega$ es denso en $L^p(\Omega)$.
\end{prop}

\begin{prop}\label{integralnula}
  Sea $f\in L^1(\Omega)$. Si 
  \begin{equation*}
    \int_\Omega f \varphi =0,
  \end{equation*}
  para toda $\varphi\in C_c(\Omega)$, entonces $f(x)=0$ para casi todo $x\in \Omega$.
\end{prop}

\begin{obs}
  Nótese que el hecho de que una sucesión $(f_n)$ converga una función $f$ en casi todo punto no implica que lo haga en $L^p(\Omega)$. Por ejemplo podemos tomar la sucesión 
  \begin{equation*}
    f_n=\frac{1}{n}\chi_{[0,n]}.
  \end{equation*}
  Esta sucesión converge a $0$ en casi todo punto, pero $\norm{f_n}_p=1$ para todo $n\in \NN$.
\end{obs}
  
\begin{proof}
  Si $f$ es continua, el resultado es fácil. Si hay algún punto $x_0 \in \Omega$ tal que $|f(x_0)| > 0$, entonces existe un $\varepsilon >0$ tal que $|f(x)|>0$ para todo $x\in B_{\varepsilon}(x_0)$. Basta tomar entonces una función $\varphi$ que valga $1$ en $B_{\varepsilon/2}(x_0)$ y $0$ fuera de $B_{\varepsilon}(x_0)$, de modo que 
  \begin{equation*}
    \left|\int_\Omega f \varphi\right| = \left|\int_{B_{\varepsilon/2}(x_0)}f\right| > 0.
  \end{equation*}

  En el caso general, en que $f$ puede no ser continua, vamos a usar la Proposición~\ref{densidad1}. Sea $\eps>0$. Sé que existe una sucesión $(f_n)$ de funciones $f_n \in C_c(\Omega)$ tal que $f_n \rightarrow f \ L_1$, es decir, tal que existe un $n_0 \in \NN$ de modo que si $n\geq n_0$ entonces $\norm{f_n-f}_1 < \eps$. Ahora, para cualquier $\varphi \in C_c(\Omega)$, se tiene
  \begin{equation*}
    \int_{\Omega} f_n \varphi = \int_{\Omega} (f_n-f) \varphi + \int_{\Omega} f \varphi = \int_{\Omega} (f_n-f) \varphi,
  \end{equation*}
  ya que $\int_{\Omega} f \varphi = 0$ por hipótesis. Ahora, si aplico la desigualdad de Hölder, 
  \begin{equation*}
    \int_{\Omega} f_n \varphi=\int_{\Omega} (f_n-f) \varphi \leq \norm{f_n-f}_1 \norm{\varphi}_{\infty} < \norm{\varphi}_{\infty} \eps.
  \end{equation*}

  Defino entonces los conjuntos
  \begin{align*}
    K_1 & = \left\{ x\in \Omega | f(x) \geq \eps \right\}, \\
    K_2 & = \left\{ x\in \Omega | f(x) \leq -\eps \right\},
  \end{align*}
  y llamo $K=K_1 \cup K_2$. Tomo además una función $\varphi \in C_c(\Omega)$ que valga $1$ en $K_1$ y $-1$ en $K_2$. Entonces
  \begin{equation*}
    \int_{\Omega} f_n \varphi = \int_K f_n \varphi + \int_{\Omega-K} f_n \varphi = \int_K |f_n| + \int_{\Omega-K} f_n \varphi.
  \end{equation*}
  Recordemos además que $\int_\Omega f_n \varphi \leq \norm{\varphi}_{\infty} \eps$. Tenemos entonces
  \begin{equation*}
    \int_K |f_n| \leq \norm{\varphi}_{\infty} \eps - \int_{\Omega-K}f_n\varphi\leq \norm{\varphi}_{\infty} \eps + \eps \norm{\varphi}_{\infty} \mu(\Omega-K) \leq \cte \eps.
  \end{equation*}
  Por tanto,
  \begin{equation*}
    \int_\Omega |f_n| \leq \int_K |f_n| + \int_{\Omega-K}|f_n| \leq \cte \eps + \eps \mu(\Omega-K) = \cte \eps.
  \end{equation*}

  De aquí concluimos que 
  \begin{equation*}
    \lim_{n\rightarrow \infty} \int_\Omega |f_n| = 0.
  \end{equation*}
  Usando el lema de Fatou, obtenemos 
  \begin{equation*}
    \int_\Omega |f| =0,
  \end{equation*}
  que implica que $f$ es nula en casi todo punto de $\Omega$, tal y como queríamos probar.
\end{proof}

\section{Derivada débil y espacios de Sobolev}
\begin{ejemplo}
  Volvamos al caso del Ejemplo~\ref{ejemplo1}. Habíamos llegado a la conclusión de que podíamos entender como una solución de nuestro problema de contorno~\eqref{eq:ejemplo2} una función $y$ tal que
  \begin{equation*}
    \int_0^1 y'\varphi'+2 \int_0^1 y \varphi = \int_0^1 f \varphi,
  \end{equation*}
  para funciones $\varphi$ «tan buenas como queramos». Queríamos entonces ver qué funciones $y$ podrían hacer que estas integrales tuvieran sentido. Para que la integral $\int_0^1 y \varphi$ tenga sentido basta pedir que $y\in L^2$. Sin embargo, ¿qué le tenemos que pedir a $y$ para que $\int_0^1 y'\varphi'$ tenga sentido? ¿Qué es $y'$? ¿Se puede definir algún tipo de \emph{derivada}?

  Más en general, si paraa una función $\varphi$, denotamos por $\varphi_{x_i}$ a la derivada parcial $\frac{\partial \varphi}{\partial x_i}$, dada una función $u$, queremos hallar una función «derivada débil» (respecto de $x_i$), que podremos denotar como $u_{x_i}$ que cumpla la fórmula de integración por partes. Es decir, $u_{x_i}$ ha de ser tal que
  \begin{equation*}
    \int u \varphi_{x_i}=-\int u_{x_i} \varphi,
  \end{equation*}
  para cualquier $\varphi$ «suficientemente buena» y de soporte compacto. De nuevo, para que esta integral tenga sentido, es necesario que esta $u_{x_i}\in L^2$. \qed
\end{ejemplo}

\begin{defn}
  Sea $u\in L^p(\Omega)$. Se dice que una función $v\in L^p(\Omega)$ para algún $p\geq 1$ es una \emph{derivada débil de $u$ respecto de $x_i$} si 
  \begin{equation*}
    \int_\Omega u \varphi_{x_i}= -\int_{\Omega} v \varphi
  \end{equation*}
  para toda $\varphi\in C^{\infty}_c(\Omega)$ (función suave con soporte compacto en $\Omega$). Esta $v$ se denota como $u_{x_i}$.

  Más generalmente, si $\alpha=(\alpha_{i_1},\dots,\alpha_{i_r})$, con $1\leq i_1 <\cdots < i_r \leq N$, se dice que una función $v\in L^p(\Omega)$ para algún $p\geq 1$ es una \emph{derivada débil de $u$ respecto del multiíndice $\alpha$}  si
  \begin{equation*}
    \int_{\Omega} u D^{\alpha} \varphi = (-1)^{|\alpha|} \int_{\Omega} v \varphi
  \end{equation*}
  para toda $\varphi \in C^{\infty}_c(\Omega)$, donde 
  \begin{equation*}
    D^{\alpha}=\frac{\partial^{\alpha_{i_1}+\cdots +\alpha_{i_r}}}{\partial x_{i_1}^{\alpha_{i_1}}\cdots \partial x_{i_r}^{\alpha_{i_r}}}.
  \end{equation*}
  Esta $v$ se denota como $D^\alpha u$.

  Finalmente, se definen los \emph{espacios de Sobolev} como los conjuntos
  \begin{equation*}
    W^{k,p}(\Omega)=\left\{ u \in L^p(\Omega): D^j u \in L^p(\Omega), \text{ para todo } |j|\leq k  \right\}.
  \end{equation*}
  En particular nos interesa el espacio
  \begin{equation*}
    H^1(\Omega)=W^{1,2}(\Omega)=\left\{ u \in L^2(\Omega) : u_{x_i} \in L^2(\Omega), \text{ para todo } i=1,\dots,n \right\}.
  \end{equation*}
\end{defn}

\begin{ejemplo}
  Sea $f(x)=|x|$ para $x\in (-1,1)$. Definimos
  \begin{equation*}
    f'(x)=
    \begin{cases}
      1 & x >0, \\
      -1 & x<0.
    \end{cases}
  \end{equation*}
  Veamos que, en efecto, $f'$ es la derivada débil de $f$. Basta comprobar la fórmula: dada $\varphi \in C^{\infty}_c((0,1))$, tenemos que
  \begin{align*}
    \int_{-1}^1 f(x) \varphi'(x) dx & = \int_{-1}^1 |x| \varphi'(x) dx = \int_{-1}^0 -x\varphi'(x) dx + \int_0^1 x \varphi'(x) dx \\ & = \int_{-1}^0 \varphi(x) dx - \int_0^1 \varphi(x) dx = -\int_{-1}^1 f'(x) \varphi(x) dx.
  \end{align*}
  \qed
\end{ejemplo}

  \begin{obs}
    Veremos que la condición de que las $\varphi\in C_c^{\infty}(\Omega)$ se puede relajar. Podemos considerar simplemente funciones en espacios de Sobolev y tales que «valgan 0 en el borde», más tarde precisaremos qué quiere decir esto.
  \end{obs}		
  Cabe preguntarse ahora sobre la unicidad de la derivada débil.
  \begin{prop}
    La derivada débil es única (en casi todo punto) si $\Omega$ es acotado.
  \end{prop}
  \begin{proof}
    Supongamos que existen $v_1,v_2 \in L^p(\Omega)$ tales que
    \begin{equation*}
      \int_\Omega u D^{\alpha} \varphi = (-1)^{|\alpha|}\int_\Omega v_1 \varphi=(-1)^{|\alpha|}\int_\Omega v_2 \varphi
    \end{equation*}
    para toda $\varphi \in C^{\infty}_c(\Omega)$. Entonces
    \begin{equation*}
      \int_\Omega (v_1-v_2) \varphi = 0
    \end{equation*}
    para toda $\varphi$. Como $v_1,v_2 \in L^p(\Omega)$, $v_1-v_2 \in L^p(\Omega)\subset L^1(\Omega)$, ya que $\Omega$ es acotado. Ahora, por la Proposición~\ref{integralnula}, $(v_1-v_2)(x)=0$ para casi todo $x$, luego $v_1(x)=v_2(x)$ para casi todo $x$.
  \end{proof}

  \begin{ejemplo}
    Consideremos la función
    \begin{equation*}
      u(x)=
      \begin{cases}
      x & x \in (0,1] \\
      1 & x \in (1,2).
      \end{cases}
    \end{equation*}
    Como la derivada débil es única, en la parte donde la función es derivable, debe coincidir con la derivada «clásica», por tanto, nuestra única candidata a derivada débil es la función
    \begin{equation*}
      u'(x)=
      \begin{cases}
      1 & x \in (0,1) \\
      0 & x \in (1,2).
      \end{cases}
    \end{equation*}
    En efecto, esta función es la derivada débil, ya que, dada $\varphi\in C^{\infty}_c ((0,2))$,
    \begin{equation*}
      \int_0^2 u(x) \varphi'(x)=\int_0^1 x \varphi'(x) dx + \int_1^2 \varphi'(x)dx = \varphi(1)-\varphi(1)-\int_0^1 \varphi(x) dx = -\int_0^1 \varphi(x) dx.
    \end{equation*}

  Pero cabe preguntarse entonces qué pasa si cambiamos ligeramente la función por una que no es continua, por ejemplo
    \begin{equation*}
      u(x)=
      \begin{cases}
      x & x \in (0,1] \\
      2 & x \in (1,2).
      \end{cases}
    \end{equation*}
    En este caso, dada $\varphi \in C^{\infty}_c( (0,2))$,
    \begin{equation*}
      \int_0^2 u(x)\varphi'(x)dx=\int_0^1 x \varphi'(x)dx + \int_1^2 2\varphi'(x)dx=\varphi(1) - \int_0^1 \varphi(x) dx - 2\varphi(1) = -\int_0^1\varphi(x)dx- \varphi(1),
    \end{equation*}
    que es, en general, distinto de $-\int_0^1 \varphi(x) dx$.

    ¿Es posible encontrar entonces alguna derivada débil? Es decir, ¿existe alguna $v$ tal que 
    \begin{equation*}
      -\int_0^1 v \varphi = -\int_0^1 \varphi - \varphi(1),
    \end{equation*}
    para toda $\varphi \in C^{\infty}_c( (0,2))$?
    La respuesta es que no. Para verlo consideremos la sucesión de funciones $(\varphi_n)$, con 
    \begin{equation*}
      \varphi_n (x)=
      \begin{cases}
	0 & x \leq -1/n, \\
      1+nx & x \in (-1/n,0], \\
    1-nx & x \in (0,1/n], \\
    0 & x > 1/n.
      \end{cases}
    \end{equation*}
    (Realmente estas funciones no me sirven, porque no son $C^\infty$, pero puedo tomar unas funciones similares, «redondeadas».) Puedo tomar entonces un $n$ lo suficientemente grande de forma que
    \begin{equation*}
      -\int_0^1 v \varphi_n + \int_0^1 \varphi_n < \varphi_n(1)=1.
    \end{equation*}

    En general, este problema lo van a tener todas las funciones que no son continuas, de hecho, veremos la inclusión $H^1(\Omega) \subset C(\Omega)$. \qed
  \end{ejemplo}
\end{document}


